
\chapter{Introduction}

The \nutils{} are a set of \unix{} shell programs for working with Newick-formatted phylogenetic trees. Their main features are:
\begin{itemize}
 \item they require no user interaction.\footnote{Why this is a good thing is not the focus of this document: I shall assume that if you are reading this, you already know when a command-line interface is better than an interactive interface.}
 \item they can work on any number of trees at a time\footnote{Strictly speaking, a few applications are limited to one input tree because working on more than one is not practical}
 \item they perform reasonably well with large trees
 \item they are implemented as filters
\end{itemize}
They are not tools for \emph{making} phylogenies. Rather, they are for working with existing ones, by which I mean manipulating the tree or extracting information from it: rerooting, simplifying, extracting subtrees, printing branch lengths and distances, etc - a glance at the table of contents of this document should give you an idea.

Each of the programs performs one task (with some variants). For example, here is how you would reroot a series of phylograms contained in file \texttt{mytrees.nw} using node \texttt{Dmelano} as outgroup:

\begin{verbatim}
$ nw_reroot mytrees.nw Dmelano
\end{verbatim} 
Now, you might want to make cladograms from the rerooted trees. Program \topology{} does the job, and since the utilities are filters, you can do it in a single command:
\begin{verbatim}
$ nw_reroot mytrees.nw Dmelano | nw_topology -
\end{verbatim}
As you can see, it is straightforward to pipe \nutils{} together, and of course they can be mixed freely with any other shell tool (see e.g. \ref{sct_counting_leaves}).

This document is organized as follows: chapter \ref{chap_general} discusses common features of the \nutils, chapter \ref{chap_simple} shows examples of simple tasks, and chapter \ref{chap_adv} has examples of more advanced tasks. 

