\section{Ordering Nodes}
\label{sct_ordering}

Two trees that differ only by the order of children nodes within the parent
convey the same biological information, even if the text (\nw) and graphical
representations differ.  For example, files \texttt{falconiformes} and
\texttt{falconiformes\_2} are different, and they yield different graphs:

\verbatiminput{order_1_svg.cmd}
\begin{center}
\includegraphics{order_1_svg.pdf}
\end{center}

\verbatiminput{order_2_svg.cmd}
\begin{center}
\includegraphics{order_2_svg.pdf}
\end{center}

\noindent{}But do they represent different phylogenies? In other words, do they
differ by more than just the ordering of nodes? To check this, we pass them to
\order:

\verbatiminput{order_3_txt.cmd}
\verbatiminput{order_3_txt.out}
