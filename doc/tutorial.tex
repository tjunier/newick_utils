\documentclass[a4paper,10pt]{report}


% Title Page
\title{Newick Utilities Tutorial}
\author{Thomas Junier}

\newcommand{\nutils}{Newick Utilities}
\newcommand{\unix}{\textsc{Unix}}
\newcommand{\ascii}{\textsc{ASCII}}
\newcommand{\svg}{\textsc{SVG}}
\newcommand{\display}{\texttt{nw\_display}}

\begin{document}
\maketitle
\tableofcontents

\chapter{Introduction}

The \nutils{} are a set of \unix{} shell programs for working with Newick-formatted phylogenetic trees. Their main feature is that they require no user interaction. Why this is a good thing is not the focus of this document: I shall assume that if you are reading this, you already know when a command-line interface is better than an interactive interface.

The \nutils{} are not tools for \emph{making} phylogenies. Rather, they are for working with existing ones, by which I mean manipulating the tree or extracting information from it. Each of the program performs one task (with some variants). For example, here is how you would reroot a series of trees contained in file \texttt{mytrees.nw} using node \texttt{Dmelano} as outgroup:
\begin{verbatim}
$ nw_reroot mytrees.nw Dmelano
\end{verbatim} 
Now, you might want to make cladograms from the rerooted trees. This would be done like this:
\begin{verbatim}
$ nw_reroot mytrees.nw Dmelano | nw_topology -
\end{verbatim}

This document is organized as follows: chapter \ref{chap_general} discusses common features of the \nutils, chapter \ref{chap_simple} shows examples of simple tasks, and chapter \ref{chap_adv} has examples of more advanced tasks.

\section{Conventions}

Most of the utilities produce Newick output. This is a rather terse format, not easily readable for humans. Therefore, I will generally show the result in graph form, either as \ascii{} or as (rendered!) \svg. The conversion from Newick to graph will be performed with \display{}. To avoid tedious repetitions, however, I will not always explicitly show the \display{} step; that is, if I say that 
\begin{verbatim}
$ nw_clade catarrhini.nw Pongo Pan
\end{verbatim} 
results in this:
\begin{samepage}
\begin{verbatim}
                  +-----------------+ Gorilla     
                  |                               
 +----------------+ Homininae +----------+ Pan    
 |                +-----------+ Hominini          
=| Hominidae                  +----------+ Homo   
 |                                                
 +---------------------------------+ Pongo  
\end{verbatim} 
\end{samepage}
it will mean that the last step was \display{}, something like this:
\begin{verbatim}
$ nw_clade catarrhini.nw Pongo Pan | nw_display -
\end{verbatim}
If the graph is \svg, then it will have been \display{} \verb+-s+.
\chapter{Generalities}

\label{chap_general}
 
\chapter{Simple Tasks}
\label{chap_simple}

\section{Rerooting}

\section{Labels}

\chapter{Advanced Tasks}
\label{chap_adv}
\end{document}
