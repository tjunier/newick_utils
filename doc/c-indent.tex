%&context

\section[sct_indent]{Indenting}


\nwindent{} reformats \nw{} on several lines, with one node per line,
nodes of the same depth in the same column, and children nodes to the right of
their parent. This shows the structure more
clearly than the compact form, but since whitespace is ignored in the Newick
format\footnote{except between quotes}, the
indented form is still valid. For example, this is a tree in compact form, in
file \filename{falconiformes}:

\typefile{falconiformes.nw}

And this is the same tree, indented:
\nwCmdOutput{indent_1}

The structure is much more clear, it is also relatively easy to edit manually
in a text editor - while still being valid \nw.

Another advantage of indenting is that it is resistant to certain errors
which would cause \display{} to fail.\footnote{This is
because indenting is a purely lexical process, hence it does not need a
syntactically correct tree.} For example, there is an error in this tree:
\typefile{falconiformes_error.nw}
yet it is hard to spot, and trying \display{} won't help as it will abort with a
parse error. With \nwindent{}, however, you can at least look at the tree:

While the error is not exactly obvious, you can at least view the Newick. It turns out there is a comma missing after \code{Sagittarius:5}.

The indentation can be varied by supplying a string (option \code{-t}) that
will be used instead of the default (which is two spaces). If you want to
indent by four spaces instead of two, you could say this:
\nwCmdOutput{indent_2}

Option \code{-t} can also be used to highlight indentation:
\txtCmdOutput{indent_3}

Now the indentation levels are easier to see, but at the expense of the tree no
longer being valid \nw.

Finally, option \code{-c} ("compact") does the reverse: it removes all
indentation and produces a compact tree. You can use this when you want to
produce a compact Newick file after editing. For example, using Vim, after
loading a \nw{} tree I do 

\startnarrower
\type+gg!}nw_indent -+
\stopnarrower

to indent the file, then I edit it, then compact it again:

\startnarrower
\type+gg!}nw_indent -c -+
\stopnarrower
