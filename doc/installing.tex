\chapter{Installing the \nutils}

\section{From source}

I have tested the \nutils{} on various distributions of Linux, as well as on
Mac OS X\footnote{I use Linux as a main development platform. Although I try my
best to get the package to compile on Macs, I don't always succeed.}. On Linux,
chances are you already have development tools preinstalled, but some
distributions (\eg, Ubuntu) do not install \textsc{gcc}, etc. by default. Check
that you have \textsc{gcc}, Bison, Flex, and the \textsc{gnu} autotools,
including Libtool. On MacOS X, you need to install XCode
(\url{http://developer.apple.com/tools/xcode}).

The package uses the \textsc{gnu} autotools, like many other open source
software packages. So all you need to do is the usual
\begin{verbatim}
$ tar xzf newick-utils-x.y.z.tar.gz
$ cd newick-utils-x.y.z
$ ./configure
$ make
$ make check
# make install
\end{verbatim}
The \texttt{make check} is optional, but you should try it anyway. Note that
the \gen{} test may fail - this is due to differences in random number
generators, as far as I can tell.


\section{As binaries}

Since version 1.1, there is also a binary distribution. The name of the archive
matches \texttt{newick-utils-<version>-<platform>.tar.gz}. All you need to do is:

\begin{verbatim}
$ tar xzf newick-utils-<vesion>-<platform>.tar.gz
$ cd newick-utils-<vesion>-<platform>
\end{verbatim}

\noindent{}The binaries are in \texttt{src}. Testing may be less important than
when installing from source, but you can do it like this:

\begin{verbatim}
$ cd tests
$ for test in test*.sh; do ./$test; done 
\end{verbatim}

\noindent{}any failure will generate a \texttt{FAIL} message (which you could filter with \texttt{grep}, etc).

You can then copy/move the binaries wherever it suits you.

