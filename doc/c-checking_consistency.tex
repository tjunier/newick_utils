%&context

\section[sct_check_consistency]{Checking Consistency with other Data}


\subsection{By condensing}

One can check the consistency of a tree with respect to additional information
by renaming and condensing. For example, I have the following tree of
Falconiformes (diurnal raptors: eagles, falcons, etc):

\externalfigure[checking_consistency_1_svg]

Now I also have the following information about the family to which each genus belongs:

\starttabulate[|l|l|]
\NC {\bf Genus} \NC {\bf Family} \NC\NR
\NC \sciname{Accipiter} \NC Accipitridae \NC\NR
\NC \sciname{Aquila} \NC Accipitridae \NC\NR
\NC \sciname{Buteo} \NC Accipitridae \NC\NR
\NC \sciname{Elanus} \NC Accipitridae \NC\NR
\NC \sciname{Falco} \NC Falconidae \NC\NR
\NC \sciname{Haliaeetus} \NC Accipitridae \NC\NR
\NC \sciname{Micrastur} \NC Falconidae \NC\NR
\NC \sciname{Milvago} \NC Falconidae \NC\NR
\NC \sciname{Milvus} \NC Accipitridae \NC\NR
\NC \sciname{Pandion} \NC Pandionidae \NC\NR
\NC \sciname{Polyborus} \NC Falconidae \NC\NR
\NC \sciname{Sagittarius} \NC Sagittariidae
\stoptabulate

Let's see if the tree is consistent with this information. If it is, all
families should form clades. To check this, I will rename each leaf by replacing
the genus name by the family name, then condense the tree. If the original tree
is consistent, the final tree should have one leaf per family.

First, I create a renaming map (see \in{}[sct_rename]) based on the above
information (here are the first three lines):
\txtCmdOutput{checking_consistency_2}
Then I use it to rename, and then I condense the tree:

\svgCmdOutput{checking_consistency_3}

As we can see, there is one leaf per family, so the above information is
consistent with the tree.  Let's see if common English names are also consistent
with the tree. Here is one possible table of vernacular names of the raptor
genera: 

\starttabulate[|l|l|]
\NC {\bf Genus} \NC {\bf English name} \NC\NR
\NC \sciname{Accipiter} \NC hawk (sparrowhawk, goshawk, etc)\NC\NR
\NC \sciname{Aquila} \NC eagle  \NC\NR
\NC \sciname{Buteo} \NC hawk \NC\NR
\NC \sciname{Elanus} \NC kite \NC\NR
\NC \sciname{Falco} \NC falcon \NC\NR
\NC \sciname{Haliaeetus} \NC eagle (sea eagle) \NC\NR
\NC \sciname{Micrastur} \NC falcon (forest falcon)\NC\NR
\NC \sciname{Milvago} \NC caracara \NC\NR
\NC \sciname{Milvus} \NC kite \NC\NR
\NC \sciname{Pandion} \NC osprey \NC\NR
\NC \sciname{Polyborus} \NC caracara \NC\NR
\NC \sciname{Sagittarius} \NC secretary bird
\stoptabulate

And here is the corresponding tree:

\svgCmdOutput{checking_consistency_4}

So the above common names are consistent with the tree. However, some species
have many common names. For example, the \sciname{Buteo} hawks are often called
"buzzards" (in Europe), and two species of falcons have been called "hawks" (in
North America): the peregrine falcon (\sciname{Falco peregrinus}) was called the
"duck hawk", and the American kestrel (\sciname{Falco sparverius}) was called
the "sparrow hawk".\footnote{This is confusing because there are true hawks
called "sparrow hawks", {\em e.g.} the Eurasian sparrow hawk \sciname{Accipiter
nisus}. To add to the confusion, the specific name \sciname{sparverius} looks
like the English word "sparrow", and also resembles the common name of
\sciname{Accipiter nisus} in many other languages: \foreign{\'{e}pervier} (fr),
\foreign{Sperber} (de), \foreign{sparviere} (it).  Oh well. Let's not drop
scientific names just yet!} If we map these common names to the tree and
condense, we get this:

\svgCmdOutput{checking_consistency_5}

Distinguishing buzzards from other hawks fits well with the tree. On
the other hand, calling a falcon a hawk does not, hence the name "hawk" appears
in two different places.
