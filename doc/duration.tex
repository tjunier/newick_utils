\section{Converting
Node Ages to Durations}
\label{sct_duration}

Sometimes you have information about the \textit{age} of a node rather than the
length of its branch. Consider the following phylogeny of major chordate groups:

\begin{center}
 \includegraphics{duration_1_svg.pdf}
\end{center}

Suppose we have the following information about the age of certain events (not
that it matters, I found it in Wikipedia and the Palaeos website
(\url{www.palaeos.com}):

\smallskip
\begin{tabular}{lr}
\textbf{event} & \textbf{age} (million years ago)\\
\hline
split of vertebrates into gnathostomes and conodonts & 530 \\
extinction of conodonts & 200 \\
split of chordates into vertebrates and urochordates & 540 \\
\end{tabular}
\smallskip

\noindent{}We can use the branch length field of Newick to specify ages, like
this:

\begin{samepage}
\verbatiminput{duration_2_nw.cmd}
\verbatiminput{duration_2_nw.out}
\end{samepage}

\noindent{}The "branch length" of \texttt{Vertebrata} becomes \texttt{530},
because the vertebrate lineage split into conodonts and gnathostomes at that
date\footnote{Of course there are other branches in the vertebrate lineage, but
they are not shown in this tree}. Note that the \texttt{Gnathostomata} leaf has
no age: this means that there are still living gnathostomes (such as you and
I\footnote{Well, at least \textit{I} am one}); the same goes for urochordates.
In other words, a leaf with no age has an implicit age of zero. This also
ensures that the leaves of the extant taxa are aligned.  The \texttt{Conodonta},
on the other hand, has an age although it is a leaf: this is because the
conodonts went extinct, around \texttt{200} Mya.

Now, if we were to display this tree without further ado, it would be nonsense.
We have to convert the ages into durations, and this is the function of
\duration{}:

\verbatiminput{duration_3_svg.cmd}
\begin{center}
 \includegraphics{duration_3_svg.pdf}
\end{center}

\noindent{}We can improve the graph by supplying option \texttt{-t} to
\display{}: this aligns the origin of the scale bar with the leaves and counts
backwards. To top it off, we'll specify the units as million years ago:

\verbatiminput{duration_4_svg.cmd}
\begin{center}
 \includegraphics{duration_4_svg.pdf}
\end{center}


Since you're curious, here is what the \texttt{age.nw} tree looks like if we
"forget" to run it through \duration:

\verbatiminput{duration_5_svg.cmd}
\begin{center}
 \includegraphics{duration_5_svg.pdf}
\end{center}

\noindent{}Now it looks as though only the conodonts are still alive, while the
gnathostomes and urochordates each had brief flashes of existence 200 and 730
million years ago, respectively. Don't show this to a palaeontologist.


