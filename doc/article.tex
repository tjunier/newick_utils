\documentclass[a4paper,10pt]{article}
\usepackage{url}

%opening
\title{UNIX Shell Utilities for Newick Trees}
\author{Thomas Junier}

\begin{document}

\maketitle

\begin{abstract}
The Newick Utilities are a set of UNIX command-line programs for working with
Newick trees.
\end{abstract}
\section{Introduction}

There is a vast choice of programs for manipulating phylogenetic trees (e.g.
NJPlot \cite{Raton2008} or TreeView, more examples at \url{
http://evolution.genetics.washington.edu/phylip/software.html}), which allow
operations like (re)rooting, reordering nodes, editing labels, extracting
subtrees, etc). Virtually all of these applications have an interactive,
graphical interface. While this makes the tools user-friendly, it also makes it
awkward to work on large numbers of trees, or to work on very large trees; it
also makes it hard or impossible to automate operations. Faced with the need to
manipulate large or numerous trees in automated pipelines, we designed and wrote
a series of programs that need no user interaction and work with any number of
trees of any size, the sole limit being the computer's resources. The results
are the subject of this paper.

\section{Design}

\begin{itemize}
 \item one tool for one operation
 \item no user interaction required once the program runs: programs can be
 integrated in pipelines and shell scripts, or called from other programs.
 \item implement as filters: programs should accept data on standard input, and print results on standard output. This makes it easy to build pipelines and allows interaction with other UNIX shell tools.
 \item avoid menus, fixed filenames, and control files: all parameters and options are passed on the command line, all data is in files named on the command line, or in standard output.
\end{itemize}


For example, suppose we produced around 1,000 unrooted, maximum-likelihood trees
of 1:1 protein orthologs in 30 vertebrate species. We are interested in
evolution rates, so we want to extract branch lengths and submit them to
multivariate analysis. We must first root all the
trees using the ``fishes'' as outgroup. This can be done with
the following command:
\begin{quote}
\verb|$ nw_reroot vertebrata.nw Danio Fugu|                                     
\end{quote} which will reroot all Newick trees in file
\verb|vertebrata.nw| on the clade defined by \verb|Danio| and \verb|Fugu|, and
print the rerooted trees to standard output, also in Newick format. To extract
branch lengths, we pipe the rooted trees into \verb|nw_distance|:
\begin{quote}
\verb%$ nw_reroot vertebrata.nw Danio Fugu | nw_distance -t -%
\end{quote}
where option \verb|-t| request tab-separated output, and the '\verb|-|' means to
read data on standard input.

\bibliographystyle{alpha} 
\bibliography{article}


\end{document}
