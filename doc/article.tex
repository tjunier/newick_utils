\documentclass[a4paper,10pt]{article}
\usepackage{url}

%opening
\title{UNIX Shell Filters for Newick Trees}
\author{Thomas Junier}

\begin{document}

\maketitle

\begin{abstract}
The Newick Utilities are a set of UNIX command-line programs for working with
Newick trees.
\end{abstract}
\section{Introduction}

There is a vast choice of programs for manipulating phylogenetic trees (e.g.
NJPlot or TreeView, more examples at \url{
http://evolution.genetics.washington.edu/phylip/software.html}), which allow
operations like (re)rooting, reordering nodes, editing labels, extracting
subtrees, etc). Most of these applications have an interactive, graphical interface (GUI). The pros and cons of GUIs versus command-line interfaces (CLIs) have been disputed for decades, they are not the focus of this paper, but let us remark that i) interactive interfaces make it
awkward to work with large numbers of trees, or on very large trees; and ii) interactive programs are hard to integrate into other programs (be they simple shell scripts or complex C applications).

In a UNIX shell environment, filters are particularly useful because they can be strung into shell pipelines. Programs that do not behave like this (e.g. because of fixed input/output filenames or control files) require additional work (e.g. wrappers that rename input/output files or generate control files) if they are to be used in larger programs (for discussion of the advantages of filters and CLIs in general, see e.g. \cite{Gancarz2002}.

Faced with the need to
manipulate large or numerous trees in automated pipelines, we designed and wrote
a series of programs that need no user interaction and work with any number of
trees of any size, and function as filters.

\section{Functionalities}

We needed programs to do the following:

\begin{itemize}
 \item reroot trees on outgroups defined by labels
 \item extract subtrees defined by labels
 \item extract branch lengths in various ways (from root, from parent, as matrix, etc.)
 \item extract labels (leaf, inner nodes, or both)
 \item attribute support values to a known tree based on bootstrap replicates
 \item rename node labels based on a mapping
 \item order tree nodes (without changing topology)
 \item extract topological information (by discarding branch length data, etc)
 \item simplify trees (e.g. whole clades of the same label)
 \item discard nodes by label
 \item \ldots [maybe keep only the most interesting]
\end{itemize}

To simplify work in the shell, a program for displaying tree graphs as text was added to the list (it can also output SVG).

\begin{itemize}
 \item one tool for one operation
 \item no user interaction required once the program runs: programs can be
 integrated in pipelines and shell scripts, or called from other programs.
 \item implement as filters: programs should accept data on standard input, and print results on standard output. This makes it easy to build pipelines and allows interaction with other UNIX shell tools.
 \item avoid menus, fixed filenames, and control files: all parameters and options are passed on the command line, all data is in files named on the command line, or in standard output.
\end{itemize}


For example, suppose we produced around 1,000 unrooted, maximum-likelihood trees
of 1:1 protein orthologs in 30 vertebrate species. We are interested in
evolution rates, so we want to extract branch lengths and submit them to
multivariate analysis. We must first root all the
trees using the ``fishes'' as outgroup. This can be done with
the following command:
\begin{quote}
\verb|$ nw_reroot vertebrata.nw Danio Fugu|                                     
\end{quote} which will reroot all Newick trees in file
\verb|vertebrata.nw| on the clade defined by \verb|Danio| and \verb|Fugu|, and
print the rerooted trees to standard output, also in Newick format. To extract
branch lengths, we pipe the rooted trees into \verb|nw_distance|:
\begin{quote}
\verb%$ nw_reroot vertebrata.nw Danio Fugu | nw_distance -t -%
\end{quote}
where option \verb|-t| request tab-separated output, and the '\verb|-|' means to
read data on standard input.

\bibliographystyle{alpha} 
\bibliography{article}


\end{document}
